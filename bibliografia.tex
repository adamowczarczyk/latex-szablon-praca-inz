\cleardoublepage
\phantomsection
\addcontentsline{toc}{chapter}{Bibliografia}


\begin{thebibliography}{99}
%Literatura:

  \bibitem{przykladowy-obrazek}
  G. Skornowicz,
  \emph{Twórczość własna},
  wyd V, wyd. Helion, 
  Wrocław 2019
  
%przykladowy wpis:
  \bibitem{linux-security}
  M. Rash,
  \emph{Bezpieczeństwo sieci w linuksie},
  Wyd I, wyd. Helion,
  Gliwice 2008
  
%przykladowy wpis:
 \bibitem{linux-server}
	C. Binnie
	\emph{Linux Server: Bezpieczeństwo i ochrona sieci},
	Wyd I, wyd. Helion,
	Gliwice 2017
	
\end{thebibliography}


% Opis bibliograficzny wydawnictwa zwartego (książki) składa się z następujących pozycji [7]:
% autorzy (nazwisko + inicjały imion), tytuł (kursywa bez cudzysłowu), nazwa wydawnictwa,
% miejsce wydania, rok wydania (w nawiasach). Poszczególne części opisu powinny być
% oddzielone przecinkami. Przy dużej liczbie autorów można podać dane pierwszego autora z frazą
% „et al.” [5].
% 
% Opis artykułu w czasopiśmie [2,9]: autorzy (nazwisko + inicjały imion), tytuł (kursywa bez
% cudzysłowu), nazwa czasopisma, wolumin, numer, rok wydania (w nawiasach), strony „od–do”
% przedzielone znakiem półpauzy (Alt+0150), bez spacji w środku.
% 
% Opis referatu w materiałach konferencyjnych lub rozdziału pracy zbiorowej [4]: autorzy
% referatu (nazwisko + inicjały imion), tytuł referatu (kursywa bez cudzysłowu), słowo (w:),
% redaktorzy pracy zbiorowej (nazwisko + inicjały imion), słowo (red.), tytuł pracy zbiorowej lub
% dane konferencji (czcionka prosta), wydawnictwo, miejsce wydania, rok wydania (w nawiasach),
% strony „od–do” przedzielone znakiem półpauzy (Alt+0150), bez spacji w środku. Jeżeli
% opisywana praca jest częścią serii wydawniczej, należy dodać jej nazwę oraz numer woluminu
% pomiędzy tytułem pracy zbiorowej i nazwą wydawnictwa [6]. Możliwe jest także zastosowanie
% skróconego opisu referatu w materiałach konferencyjnych – bez podawania redaktorów i tytułu
% pracy zbiorowej [1].
% 
% W opisie materiałów publikowanych elektronicznie (np. specyfikacji, dokumentacji
% technicznej) należy umieścić dane autora lub nazwę producenta (jeśli nie ma podanego autora),
% tytuł dokumentu, ewentualnie opis rodzaju dokumentu (np. podręcznik użytkownika), wydawcę
% i rok wydania (w nawiasach) [8]. Opis strony internetowej składa się z danych autora albo nazwy
% organizacji publikującej, tytułu dokumentu albo nazwy serwisu, jego adresu URL (bez
% podkreślenia i niebieskiego koloru), roku opublikowania oraz daty odczytania dokumentu4 [3].
% 
% 1. Agrawal R., Srikant R., Fast Algorithms for Mining Association Rules, Proceedings
% of the Twentieth International Conference on Very Large Databases, Santiago, Chile
% (1994)
% 2. Bacchus F., Grove A., Halpern J., Koller D., From statistical knowledge bases
% to degrees of belief, Artificial Intelligence, Vol. 87, No. 1–2 (1996) 75–143
% 3. Brown D., A Beginners Guide to UML. Part 1–2., Dunstan Thomas Consulting,
% http://consulting.dthomas.co.uk (2002), (odczytano 10 września 2008 r.)
% 4. Deogun J., Jiang L., Comparative Evaluation on Concept Approximation Approaches,
% (w:) Kwaśnicka H., Paprzycki M. (red.), Proceedings of the Fifth International
% Conference on Intelligent Systems Design and Applications, IEEE Computer Society
% Press, Washington, Brussels, Tokyo (2005) 438–443
% 5. Fagin R. et al., Reasoning About Knowledge, MIT Press, Cambridge, USA (1995)
% 6. Kazakov D., Kudenko D., Machine Learning and Inductive Logic Programming for
% Multi-Agent Systems, (w:) Luck M., Marik V., Stepankova O. (red.), Multi-Agent
% Systems and Applications, Lecture Notes in Artificial Intelligence (LNAI), Vol. 2086,
% Springer-Verlag, Berlin Heidelberg (2001) 246–270
% 7. Kerninghan B.W., Ritchie D.M., Język ANSI–C, Wydawnictwa Naukowo-Techniczne,
% Warszawa (1994)
% 8. Microsoft, Books On-Line, dokumentacja elektroniczna systemu MS SQL Server 2000
% Enterprise Edition, Microsoft Corporation (2000)
% 9. Perry P., Walnum C., Pierwsze kroki Telefonii GSM, Kwartalnik Elektroniki
% i Telekomunikacji, Vol. 43, No. 3 (1997) 421–430
